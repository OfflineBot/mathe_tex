

\section{Vektormultiplikation}
Das Vektormulitplikationsverfahren kann nicht wie bei der addition verlaufen. 
Dafür gibt es spezielle Methoden um eigenschaften der Vektoren zu bestimmen.

\subsection{Skalarprodukt}
Das Skalarprodukt ist eine Art einen Vektor mit einem anderen zu Multiplizieren.
Dabei werden die einzelnen Achsen der Vektoren multipliziert und anschließend summiert. 
Wenn das Skalarprodukt 0 ergibt bedeutet es, dass die Vektoren Orthogonal zueinander stehen (90°).
\\\\
Formel: \\\\
$
\vec{a} \bullet \vec{b} = |\vec{a}| \cdot |\vec{b}| \cdot cos(\alpha)
$
\\\\
\underline{Beispiel:} \\\\
$
\begin{pmatrix}
    1 \\ 
    2 \\ 
    3 \\
\end{pmatrix}
\bullet
\begin{pmatrix}
    4 \\
    5 \\
    6 \\
\end{pmatrix}
=
1 * 4 + 2 * 5 + 3 * 6
= 
32
$
\subsubsection{Schattenwurf}


\subsection{Kreuzprodukt}
$
\vec{a} = 
\begin{pmatrix}
    a_1 \\ a_2 \\ a_3
\end{pmatrix}
\vec{b} =
\begin{pmatrix}
    b_1 \\ b_2 \\ b_3
\end{pmatrix}
$
\\\\
Formel: \\
$
\vec{a}\ x\ \vec{b} = 
\begin{pmatrix}
    a_2 \cdot b_3 - a_3 \cdot b_2 \\
    a_3 \cdot b_1 - a_1 \cdot b_3 \\
    a_1 \cdot b_2 - a_2 \cdot b_1 \\
\end{pmatrix}
\\\\
|\vec{a} \ x \ \vec{b}| = |\vec{a}| \cdot |\vec{b}| \cdot sin(\alpha)
$

\section{Ebenen im Raum}
\subsection{Darstellungsarten von Ebenen im Raum}
$
\begin{pmatrix}
    a_x \\ a_y \\ a_z
\end{pmatrix}
+ s * 
\begin{pmatrix}
    b_x \\ b_y \\ b_z
\end{pmatrix}
+ t * 
\begin{pmatrix}
    c_x \\ c_y \\ c_z
\end{pmatrix}
$ 
\\\\
Dabei ist $a$ der Stützvektor und $b$ mit $c$ die Richtungsvektoren.
Wenn $b$ und $c$ gleich sind ist es eine lineare Gleichung und somit keine Ebene mehr. 
(Es ist eine Gerade im Raum).

\subsection{Spurpunkte}

\subsection{Betrag eines Vektor berechnen}
Gegeben ist der Vektor $a$. \\
$
a = 
\begin{pmatrix}
    a \\ b \\ c \\
\end{pmatrix}
$ 
\\\\
Betrag (länge) des Vektor berechen: \\\\
$
|\sigma| = \sqrt{a^2 + b^2 + c^2}
$

\section{Ebenen im 3-dimensionalen Raum}
Eine Ebene im dreidimensionalen Raum kann beschrieben werden durch die:
\begin{itemize}
    \item Parameterform einer Ebene
    \item Normalenform einer Ebene
    \item Koordinatenform einer Ebene
\end{itemize}

\subsection{Koordinatenform einer Ebene}
Formel: \\
$
E: ax_1 + bx_2 + cx_3 = d
$
\\
Von Normalenform abgeleitet: \\
$\vec{x} \bullet \vec{n} = \vec{p} \bullet \vec{n}$ \\\\
\
Dabei gilt:
\begin{itemize}
    \item $\vec{X}$: Beliebiger Punkt
    \item $\vec{n}$ (a, b, c): Normalen Vektor $\vec{x} \bullet \vec{n}$
    \item $d$: Skalarprodukt von $\vec{p} \bullet \vec{n}$
\end{itemize}
\subsection{Normalenform einer Ebene}
Formel: \\\\
$
\begin{pmatrix}
    \begin{pmatrix}
        x_1 \\ x_2 \\ x_3
    \end{pmatrix}
    -
    \begin{pmatrix}
        p_1 \\ p_2 \\ p_3
    \end{pmatrix}
\end{pmatrix}
\bullet
\begin{pmatrix}
    n_1 \\ n_2 \\ n_3
\end{pmatrix}
$
\\\\
Dabei gilt:
\begin{itemize}
    \item $\vec{P}$: Stützvektor
    \item $\vec{N}$: Normalenvektor
    \item $\vec{X}$: Beliebiger Vektor (vorgegeben)
\end{itemize}
\ \\
Das ergebnis ist eine Skalar.
Darübr kann man informationen über den eingegebenen Vektor erfahren. 
\underline{Beispiel:} Wenn das Ergebnis = 0 ist, liegt der gegebene Vektor auf der Ebene.

\subsection{Parameterform einer Ebene}
Die Koordinatenform ist eine die Gleichung für eine Ebene im Raum. \\
\underline{Formel:} \\\\
$
E: \vec{x} =
\begin{pmatrix}
    p_1 \\ p_2 \\ p_3
\end{pmatrix}
+ s * 
\begin{pmatrix}
    x_1 \\ x_2 \\ x_3
\end{pmatrix}
+ t * 
\begin{pmatrix}
    y_1 \\ y_2 \\ y_3
\end{pmatrix}
$


