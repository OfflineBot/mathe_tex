\section{Hessesche Normalenform}

Formel: \\
$\overrightarrow{PR} \bullet \vec{n}_0$ \\
Dafür steht:
\begin{itemize}
    \item $\overrightarrow{PR}$: Betrag
    \item $\vec{n}_0$: Normalenvektor
\end{itemize} 
\
\\
Beispiel:
\\
$
\vec{n} = 
\begin{pmatrix}
    2 \\ -2 \\ 1
\end{pmatrix}
\\\\
|\vec{n}| = \sqrt{(2)^2 + (-2)^2 + (1)^2} = 3 
\\\\
\vec{n}_0 = \frac{1}{3} \cdot 
\begin{pmatrix}
    2 \\ -2 \\ 1
\end{pmatrix}
$
\\\\\\
Anderes Beispiel:\\
Gesucht: Abstand von $R(1|6|2)$\\
$
E: 
\begin{bmatrix}
    \vec{x} - 
    \begin{pmatrix}
        3 \\ 2 \\ 0
    \end{pmatrix}
\end{bmatrix}
\bullet
\begin{pmatrix}
    2 \\ -2 \\ 1
\end{pmatrix}
=0
\\
\overrightarrow{PR} =
\begin{pmatrix}
    1 - 3 \\ 6 - 2 \\ 2 - 0
\end{pmatrix}
=
\begin{pmatrix}
    -2 \\ 4 \\ 2
\end{pmatrix}
\\
\vec{n} = \frac{1}{3} = 
\begin{pmatrix}
    2 \\ -2 \\ 1
\end{pmatrix}
\\\\\\
Weil: d = \overrightarrow{PR} \bullet \vec{n}_0
\\
\rightarrow 
\begin{pmatrix}
    -2 \\ 4 \\ 2 
\end{pmatrix}
\cdot 
\begin{pmatrix}
    2 \\ -2 \\ 1
\end{pmatrix}
\cdot
\frac{1}{3}
=
\frac{1}{3} \cdot (-4 -8 +2) = -\frac{10}{3}
$
