
\section{Gegenseitige Lage von Geraden und Ebenen}
\subsection{Echt Parallel}
Normalenvektor $\vec{n}$ muss für beide Ebenen identisch sein. \\
\textrightarrow\ richtungs Vektoren sind somit gleich. \\
Stützvektor darf nicht identisch sein oder auf der jeweils anderen Ebene liegen.

\subsection{Identisch}
Normalenvektor $\vec{n}$ muss für beide Ebenen identisch sein.
\textrightarrow\ richtungs Vektoren sind somit gleich. \\
Sützvektor muss Vektor muss entweder identisch sein oder auf der jeweils anderen Ebene liegen.

\subsection{Schnittpunkt}
Normalenvektor $\vec{n}$ darf nicht identisch sein. 
\subsubsection{Beispiel}
Gegeben: \\
$
g:
\begin{pmatrix}
    x_1 \\ x_2 \\ x_3
\end{pmatrix}
=
\begin{pmatrix}
    1 \\ -1 \\ 2
\end{pmatrix}
+ t 
\cdot
\begin{pmatrix}
    1 \\ 2 \\ 3
\end{pmatrix}
$
\\\\
$
E: 2x_1 - 1x_2 + 3x_3 = 0
$
\\\\ 
Berechnen: \\
Berechne $t$ für Schnittpunkt mithilfe von einsetzen. \\
$
2(1 + 1t) - 1(-1 + 2t) + 3(2 + 3t) = 0 \\
2 + 2t + 1 - 2t + 6 + 9t = 0 \\
9 + 9t = 0 \\
9t = -9 \\
t = -1 \\
$
Einsetzen in $g$: \\
$
\vec{x}:
\begin{pmatrix}
    1 \\ -1 \\ 2 
\end{pmatrix}
+ 
(-1)
\cdot
\begin{pmatrix}
    1 \\ 2 \\ 3
\end{pmatrix}
= 
\begin{pmatrix}
    0 \\ -3 \\ -1
\end{pmatrix}
$ \\\\
Fazit: \\
$g$ schneidet $E$ im Punkt S(0$|$-3$|$-1) (Durchstoßpunkt)

\subsection{Windschief}
2 Ebenen können \underline{NICHT} zueinander Windschief sein.
Entweder Parallel (Identisch) oder haben einen Schnittpunkt.

