
\section{Gegenseitige Lage von Geraden und Ebenen}
\subsection{Echt Parallel}
$\vec{n}$ muss für beide Ebenen identisch sein. \\
\textrightarrow\ richtungs Vektoren sind somit gleich. \\
Stützvektor darf nicht identisch sein oder auf der jeweils anderen Ebene liegen.

\subsection{Identisch}
$\vec{n}$ muss für beide Ebenen identisch sein.
\textrightarrow\ richtungs Vektoren sind somit gleich. \\
Sützvektor muss Vektor muss entweder identisch sein oder auf der jeweils anderen Ebene liegen.

\subsection{Schnittpunkt}
\subsubsection{Beispiel}
Gegeben: \\
$
g:
\begin{pmatrix}
    x_1 \\ x_2 \\ x_3
\end{pmatrix}
=
\begin{pmatrix}
    1 \\ -1 \\ 2
\end{pmatrix}
+ t \cdot
\begin{pmatrix}
    1 \\ 2 \\ 3
\end{pmatrix}
$
\\
$
E: 2x_1 - 1x_2 + 3x_3 = 0
$

\subsection{Windschief}
2 Ebenen können \underline{NICHT} zueinander Windschief sein.
Entweder Parallel (Identisch) oder haben einen Schnittpunkt.
