
\section{Vektormultiplikation}
Das Vektormulitplikationsverfahren kann nicht wie bei der addition verlaufen. 
Dafür gibt es spezielle Methoden um eigenschaften der Vektoren zu bestimmen.

\subsection{Skalarprodukt}
Das Skalarprodukt ist eine Art einen Vektor mit einem anderen zu Multiplizieren.
Dabei werden die einzelnen Achsen der Vektoren multipliziert und anschließend summiert. 
Wenn das Skalarprodukt 0 ergibt bedeutet es, dass die Vektoren Orthogonal zueinander stehen (90°).
\\\\
Formel: \\\\
$
\vec{a} \bullet \vec{b} = |\vec{a}| \cdot |\vec{b}| \cdot cos(\alpha)
$
\\\\
\underline{Beispiel:} \\\\
$
\begin{pmatrix}
    1 \\ 
    2 \\ 
    3 \\
\end{pmatrix}
\bullet
\begin{pmatrix}
    4 \\
    5 \\
    6 \\
\end{pmatrix}
=
1 * 4 + 2 * 5 + 3 * 6
= 
32
$
\subsubsection{Schattenwurf}


\subsection{Kreuzprodukt}
$
\vec{a} = 
\begin{pmatrix}
    a_1 \\ a_2 \\ a_3
\end{pmatrix}
\vec{b} =
\begin{pmatrix}
    b_1 \\ b_2 \\ b_3
\end{pmatrix}
$
\\\\
Formel: \\
$
\vec{a}\ x\ \vec{b} = 
\begin{pmatrix}
    a_2 \cdot b_3 - a_3 \cdot b_2 \\
    a_3 \cdot b_1 - a_1 \cdot b_3 \\
    a_1 \cdot b_2 - a_2 \cdot b_1 \\
\end{pmatrix}
\\\\
|\vec{a} \ x \ \vec{b}| = |\vec{a}| \cdot |\vec{b}| \cdot sin(\alpha)
$


