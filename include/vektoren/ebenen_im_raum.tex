
\section{Ebenen im Raum}
\subsection{Darstellungsarten von Ebenen im Raum}
$
\begin{pmatrix}
    a_x \\ a_y \\ a_z
\end{pmatrix}
+ s * 
\begin{pmatrix}
    b_x \\ b_y \\ b_z
\end{pmatrix}
+ t * 
\begin{pmatrix}
    c_x \\ c_y \\ c_z
\end{pmatrix}
$ 
\\\\
Dabei ist $a$ der Stützvektor und $b$ mit $c$ die Richtungsvektoren.
Wenn $b$ und $c$ gleich sind ist es eine lineare Gleichung und somit keine Ebene mehr. 
(Es ist eine Gerade im Raum).

\subsection{Spurpunkte}

\subsection{Betrag eines Vektor berechnen}
Gegeben ist der Vektor $a$. \\
$
a = 
\begin{pmatrix}
    a \\ b \\ c \\
\end{pmatrix}
$ 
\\\\
Betrag (länge) des Vektor berechen: \\\\
$
|\sigma| = \sqrt{a^2 + b^2 + c^2}
$

