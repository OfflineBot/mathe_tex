
\section{Ebenen Formen (Beschreibungsarten von Ebenen)}
Eine Ebene im dreidimensionalen Raum kann beschrieben werden durch die:
\begin{itemize}
    \item Parameterform einer Ebene
    \item Normalenform einer Ebene
    \item Koordinatenform einer Ebene
\end{itemize}

\subsection{Koordinatenform einer Ebene}
Formel: \\
$
E: ax_1 + bx_2 + cx_3 = d
$
\\
Von Normalenform abgeleitet: \\
$\vec{x} \bullet \vec{n} = \vec{p} \bullet \vec{n}$ \\\\
\
Dabei gilt:
\begin{itemize}
    \item $\vec{X}$: Beliebiger Punkt
    \item $\vec{n}$ (a, b, c): Normalen Vektor $\vec{x} \bullet \vec{n}$
    \item $d$: Skalarprodukt von $\vec{p} \bullet \vec{n}$
\end{itemize}
\subsection{Normalenform einer Ebene}
Formel: \\\\
$
\begin{pmatrix}
    \begin{pmatrix}
        x_1 \\ x_2 \\ x_3
    \end{pmatrix}
    -
    \begin{pmatrix}
        p_1 \\ p_2 \\ p_3
    \end{pmatrix}
\end{pmatrix}
\bullet
\begin{pmatrix}
    n_1 \\ n_2 \\ n_3
\end{pmatrix}
$
\\\\
Dabei gilt:
\begin{itemize}
    \item $\vec{P}$: Stützvektor
    \item $\vec{N}$: Normalenvektor
    \item $\vec{X}$: Beliebiger Vektor (vorgegeben)
\end{itemize}
\ \\
Das ergebnis ist eine Skalar.
Darübr kann man informationen über den eingegebenen Vektor erfahren. 
\underline{Beispiel:} Wenn das Ergebnis = 0 ist, liegt der gegebene Vektor auf der Ebene.

\subsection{Parameterform einer Ebene}
Die Koordinatenform ist eine die Gleichung für eine Ebene im Raum. \\
\underline{Formel:} \\\\
$
E: \vec{x} =
\begin{pmatrix}
    p_1 \\ p_2 \\ p_3
\end{pmatrix}
+ s * 
\begin{pmatrix}
    x_1 \\ x_2 \\ x_3
\end{pmatrix}
+ t * 
\begin{pmatrix}
    y_1 \\ y_2 \\ y_3
\end{pmatrix}
$



